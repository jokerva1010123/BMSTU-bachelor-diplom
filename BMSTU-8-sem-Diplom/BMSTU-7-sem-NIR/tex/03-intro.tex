\section*{ВВЕДЕНИЕ}
\addcontentsline{toc}{section}{ВВЕДЕНИЕ}

В последние годы наблюдается экспоненциальный рост количества сложных документов и текстов, требующих более глубокого понимания, чтобы иметь возможность точно классифицировать тексты во многих приложениях. Многие подходы к машинному обучению достигли превосходных результатов в обработке естественного языка. Успех этих методов зависит от их способности понимать сложные модели и нелинейные связи внутри данных. Однако поиск подходящих структур, архитектур и методов классификации текста является непростой задачей. 

Цель данной рабате --- проведение краткого обзора методов классификации текста, формулирование критерий оценки классификации текста и на их основе сравнить рассмотренных методов.

В рамках выполнения работы необходимо решить следующие задачи:

\begin{itemize}[label = ---]
    \item провести анализ предметной области;
    \item провести краткий обзор существующих методов классификации текста;
    \item сравнить рассмотренных методов по преимуществам и недостаткам.
\end{itemize}