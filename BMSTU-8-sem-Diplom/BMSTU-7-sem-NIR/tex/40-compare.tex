\section{Сравнение существующих методов классификации текста}
Таким образом, рассматривая решения задачи классификации текста, можно выделить два подхода к решению: с помощью машинного обучения и глубокого обучения.

Машинное обучение включает модели: KNN, метод опорных векторов (SVM), деревья решений и случайные леса. KNN —-- это метод классификации, который легко реализовать и который адаптируется к любому типу пространства признаков. Эта модель также естественным образом обрабатывает случаи с несколькими классами\cite{140, 141}. Однако KNN ограничен ограничениями по хранению данных для больших задач поиска ближайших соседей. Кроме того, производительность KNN зависит от поиска значимой функции расстояния, что делает этот метод сильно зависимым от данных алгоритмом\cite{142, 143}. SVM —-- мощный алгоритм классификации текста, но он требует соответствующей предварительной обработки данных и может не подходить для нелинейных или сложно структурированных данных. SVM способен хорошо работать в пространствах объектов большой размерности. При работе с текстовыми данными пространство признаков часто бывает большим, поскольку количество слов или фраз может быть очень большим. SVM хорошо справляется с этим случаем и дает точные результаты классификации. Однако SVM требует предварительной обработки данных для извлечения признаков. Дерево решений — это очень быстрый алгоритм как для обучения, так и для прогнозирования, но он также чрезвычайно чувствителен к небольшим изменениям в данных\cite{166}. Случайные леса (т. е. набор деревьев решений) очень быстро обучаются по сравнению с другими методами, но довольно медленно создают прогнозы после обучения. Таким образом, чтобы добиться более быстрой структуры, количество деревьев в лесу необходимо уменьшить, поскольку большее количество деревьев в лесу увеличивает временную сложность на этапе прогнозирования.

Глубокое обучение — один из самых мощных методов искусственного интеллекта (ИИ), и многие исследователи и ученые сосредоточены на архитектурах глубокого обучения, чтобы повысить надежность и вычислительную мощность этого инструмента. Однако архитектуры глубокого обучения также имеют некоторые недостатки и ограничения при применении к задачам классификации текста. Одна из основных проблем этой модели заключается в том, что глубокое обучение не способствует всестороннему теоретическому пониманию процесса обучения\cite{202}. Хорошо известным недостатком методов глубокого обучения является их природа «черного ящика»\cite{203, 204}. То есть метод, с помощью которого методы глубокого обучения получают свернутый результат, не совсем понятен. Еще одним ограничением глубокого обучения является то, что для него обычно требуется гораздо больше данных, чем для традиционных алгоритмов машинного обучения, а это означает, что этот метод нельзя применять для задач классификации небольших наборов данных \cite{205,206}. Кроме того, огромный объем данных, необходимых для алгоритмов классификации глубокого обучения, еще больше усугубляет вычислительную сложность на этапе обучения\cite{207}.

Исходя из выше сказанно, можно сделать сравнения.

\begin{itemize}[label = ---]
    \item По сложности: Машинное обучение и глубокое обучение имеют разные сложности в процессе обучения и развертывания модели. В машинном обучении часто используются традиционные алгоритмы, такие как KNN, машины опорных векторов (SVM) или случайные леса. Для глубокого обучения модель нейронной сети будет иметь множество скрытых слоев, что создает более сложную сеть и требует больше вычислительных ресурсов.
    \item По обобщению: Глубокое обучение часто имеет лучшую способность к обобщению, чем машинное обучение. Это означает, что глубокое обучение способно изучать сложные функции и автоматически извлекать информацию из данных. Машинное обучение также может дать хорошие результаты, но оно во многом зависит от ручного выбора функций и извлечения их из данных.
    \item По требуемым данным: для достижения хороших результатов глубокое обучение часто требует большого объема обучающих данных. Благодаря машинному обучению меньшее количество обучающих данных может дать лучшие результаты. Однако, если данных достаточно, глубокое обучение может изучить более сложные закономерности и дать более точные результаты.
    \item По времени обучения: обучение глубокому обучению часто занимает больше времени, чем машинное обучение. При использовании глубокой нейронной сети процесс обучения может длиться от нескольких часов до нескольких дней и даже недель. Между тем, машинное обучение позволяет быстро обучаться и давать хорошие результаты при меньшем объеме данных.
\end{itemize}

\subsection*{Вывод}
На основе сравнения можно сделать вывод, что методы глубокого обучения требуют больше данных и времени на обучение, чем методы машинного обучения. Методы машинного обучения также имеют лучшее обобщение, чем методы глубокого обучения.






