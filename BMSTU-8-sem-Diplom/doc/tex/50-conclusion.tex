\part*{ЗАКЛЮЧЕНИЕ}
\addcontentsline{toc}{part}{ЗАКЛЮЧЕНИЕ}
В рамках настоящей работы был разработан, реализован и исследован метод классификации новостных текстов по тематикам с использованием опорных векторов. Все поставленные задачи были выполнены.

Был проведен анализ предметной области. Были рассмотрены методы очистки и предварительной обработки текста. Был проведен обзор методов извлечения признаков из текста.

Был рассмотрены и проанализированы основные методы классификации текстов. Была представлена формализованная постановка задачи в виде IDEF0-диаграммы.

Были описаны этапы работы алгоритма разрабатываемого метода, также были рассмотрены функциональные схемы обучения классификатора и метод классификации. 

Были приведены метрики, используемые для оценки качества работы обученного классификатора. Был описан используемый набор данных для обучения классификатора и была рассмотрена структура программного обеспечения.

Были приведены выбор языка программирования и средства программирования и рассмотрены необходимые библиотеки, которые используются для разработки программного обеспечения. Также был описан формат входных и выходных данных. Также были приведены описание пользовательского интерфейса и руководство пользователя для установки и использования программного обеспечения.

Были описаны технические характеристики устройства для проведения исследований и приведены исследования разработанного метода

Для реализованного метода можно предложить следующие развития:
\begin{itemize}[label = ---]
    \item ускорение работы метода.
    \item добавление возможности работы с различными языками.
\end{itemize}

