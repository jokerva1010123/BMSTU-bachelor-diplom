\part*{ВВЕДЕНИЕ}
\addcontentsline{toc}{part}{ВВЕДЕНИЕ}
В современном информационном обществе огромное количество новостей и статей публикуется каждый день. Чтобы эффективно обрабатывать и анализировать этот объем информации, важно иметь систему классификации новостей. Классификация позволяет автоматически определять тематику новости, ее важность, тональность и другие характеристики. Задача классификации новостей является актуальной и интересной задачей в области обработки естественного языка и машинного обучения. Она имеет широкий спектр применений, начиная от организации и ранжирования новостных потоков до автоматического анализа общественного мнения и выявления трендов.

Решение задачи классификации новостей может иметь большое практическое применение, включая создание интеллектуальных новостных агрегаторов, систем мониторинга общественного мнения, фильтрацию и организацию информации для пользователей. Это способствует улучшению доступа к информации, оптимизации процессов принятия решений и повышению качества новостных сервисов.

Однако, классификация новостей может быть сложной задачей из-за разнообразия тематик, стилей и тональностей новостных статей. Нередко встречаются случаи смешения нескольких тем в одной статье или субъективной интерпретации информации. Поэтому выбор правильного алгоритма и подхода, а также качество и разнообразие обучающих данных играют важную роль в достижении высокой точности классификации.

Цель работы --- разработать метод классификации новостных текстов по тематикам с использованием опорных векторов. Для достижения поставленной цели необходимо выполнить следующие задачи:
\begin{itemize}[label = ---]
    \item провести анализ предметной области, проанализировать предметную область и основные методы классификации текстов;
    \item разработать метод классификации новостных текстов по тематикам с помощью опорных векторов;
    \item разработать программное обеспечение, реализующее данный метод;
    \item оценить результаты работы метода в зависимости от различных параметров программного обеспечения.
\end{itemize}