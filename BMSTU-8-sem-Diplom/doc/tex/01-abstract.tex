\part*{РЕФЕРАТ}
\addcontentsline{toc}{part}{РЕФЕРАТ}

Расчетно--пояснительная записка содержит 67 с., \totalfigures\ рис., \totaltables\ табл., 21 ист., 1 прил.

\textbf{Ключевые слова}: метод классификации, классификации текста, машинное обучение, метод опорных векторов.

В работе представлена разработка метода классификации новостных текстов по тематикам с использованием опорных векторов.

Рассмотрена задача классификации текста. Рассмотрены этапы решения задачи и основные методы классификации текста. Проведена формализация постановки задачи в виде IDEF0-диаграммы. Разработан метод классификации новостных текстов по тематикам с использованием опорных векторов. Представлена реализация разработанного метода, приведены результаты исследования качества классификатора в зависимости от различных параметров.
